\documentclass[a4paper,12pt]{report}

\usepackage{amsmath,amsfonts,mathtools}
\usepackage{amssymb}
\usepackage{amsbsy}
\usepackage{hyperref}

\begin{document}
\title{CHE260 Part 2 Abridged}
\author{Aman Bhargava}
\date{September 2019}
\maketitle

\tableofcontents

\chapter{Basics of Heat Transfer}
\section{Introduction}
Heat is energy transferred because of temperature differences. The modes of heat transfer include:
\begin{enumerate}
\item Conduction
\item Convection
\item Radiation
\end{enumerate}

\section{Conduction}
\paragraph{Fourier's Law of Heat Conduction: }
$$\dot{Q} = -kA\frac{\Delta T}{\Delta x}$$
\begin{enumerate}
\item $k$: THERMAL CONDUCTIVITY. It's really a function of temperature but we assume it's constant in this course.
\item $A$: Area of surface
\end{enumerate}

\paragraph{DIFFUSIVITY: } $\alpha = \frac{k}{\rho c_p}$ is the rate of heat propagation through a medium by volume. Small $\to$ mostly absorbed, not transmitted.

\section{Convection}
Heat transfer from solid to fluid in motion $\dot{Q}_{conv} = \dot{Q}_{cond} + \dot{Q}_{fluid motion}$

\paragraph{NEWTON'S LAW OF COOLING: }
$$\dot{Q}_{conv} = hA_s(T_s-T_{\infty}$$
Where $h$ is experimentally and subjectively determined.

\section{Radiation}
$$\dot{Q}_{emit} = \varepsilon \sigma A_s T_s^4$$
\begin{enumerate}
\item $\varepsilon$: Emissivity
\item $\sigma$: Stefan-Boltzmann constant
\end{enumerate}
$$\dot{Q}_{rad, net} = \varepsilon \sigma A_s(T_s^4 - T_{surr}^4)$$

$$\dot{Q}_{conv+rad} = h_{combined}A_s(T_s-T_{\infty})$$

\chapter{Steady Heat Conduction}
\section{Thermal Resistance Networks}
$$\dot{Q}=\frac{T_{\infty,1} - T_{\infty,2}}{R_{total}}$$
$V = IR \to \Delta T = \dot{Q}R$

Resistances add in the same ways as electrical resistances. For a wall, you must account for convective resistances on both sides and the 
internal conductive resistance.

\paragraph{Thermal contact resistance: } Results from microscopic gaps between two objects in physical contact. $R_c$ is the resistance 
per unit area of contact. Use the ratio of $R_{layer} : R_c$ to determine if the contact resistance is worth taking into consideration.

\section{Heat Conduction with Cylinders and Spheres}
\subsection{Cylinders}
By integrating $\dot{Q}_{cond, cyl} = -kA\frac{dT}{dr}$ we get $$R_{cyl} = \frac{1}{2\pi Lk}\ln(r_2/r_1)$$
Where $r_2$ is outer, $r_1$ is inner, $L$ is length, etc.

\subsection{Spheres}
By the same logic: $$R_{sph} = \frac{r_2-r_1}{4\pi r_1 r_2 k}$$

\paragraph{For both spheres and cylinders: } you must figure in $R_{conv, in} + R_{cyl} + R_{conv, out}$

\section{Critical Radius of Insulation}
For flat walls, more insulation $\to$ more thermal resistance.

$$r_{cr,cyl} = \frac{k}{h}$$
$$r_{cr,sph} = \frac{2k}{h}$$

\section{Chapter 10 Class Notes}
\begin{itemize}
\item Biot number: Ratio of convection : conduction = $Bi = \frac{hw}{k}$, $w$ is the width of the surface.
\item Diffusion time: $t_D = \frac{L^2}{\alpha}$
\item Rate Limiting Step: If $Bi < 1$: CONVECTION. Else, conduction. 
\item Governing Conduction Equation: $$\rho C\frac{\partial T}{\partial t} = \nabla \cdot(k\nabla T) + \dot{S}$$
\begin{itemize}
\item $\rho$ = density
\item $C$ = heat capacity
\item $k$ = thermal CONDUCTIVITY
\item $\dot{S}$ = heat supplied per second
\end{itemize}
\end{itemize}

\section{Fins}
Fins allow you to increase the area of convection, leading to greater $\dot{Q}_{conv}$. We analyze fins at steady state with constant $k, h$. 

\paragraph{FIN EQUATION: } For an infinitesimal slice of a fin, we have: 
$$\dot{Q}_{cond, x} = \dot{Q}_{cond, x + \Delta x} + \dot{Q}_{conv}$$
$$\frac{d^2\theta}{dx^2} - m^2\theta = 0$$
Where $m^2 = \frac{h\rho}{kA_c}$, $\theta = T-T_\infty$
$$\theta(x) = c_1e^{mx} + c_2e^{-mx}$$
Boundary conditions: $\theta(0) = T_{base} - T_\infty$. Boundary condition 2 depends on situation...

Note that $p$ is the perimeter of the fin at the tip.

\begin{itemize}
\item \textbf{Infinite Fin Length: } $T_{tip} = T_\infty$.
\item \textbf{Insulated Tip: } $\dot{Q}_{tip} = 0 \to \frac{d\theta}{dx}|_{x=L} = 0$, so the general heat out is:
$$\dot{Q} = \sqrt{hp kA_c}(T_b-T_\infty) \tanh(mL)$$
\item Convective and radiative heat loss. Use insulated tip length with corrected length:$$L_c = L + A_c/p$$
$L_{c,rect} = L + t/2$, t = thickness. $L_{c, cyl} = L + D/4$. 

\end{itemize}

\paragraph{FIN EFFICIENCY: } $\eta_{fin} = \frac{\dot{Q}_{fin}}{\dot{Q}_{fin, max}}$. 
$$\dot{Q}_{max} = hA_{fin}(T_b-T_\infty)$$

\paragraph{FIN EFFECTIVENESS: } $$\varepsilon_{fin} = \frac{\dot{Q}_{fin}}{\dot{Q}_{nofin}} = \frac{\dot{Q}_{fin}}{hA_b(T_b-T_\infty)}$$
$$\varepsilon_{fin} = \frac{A_{fin}}{A_b}\eta_fin$$

\paragraph{Proper Fin Length: } Too long is a waste. $\frac{\dot{Q}_{fin}}{\dot{Q}_{long fin}} = tan{mL}$. Generally $mL = 1$ is a good compromise
for 76.2\% efficiency. 

$$\dot{Q}_{fin} = \frac{T_b-T_\infty}{R}$$
$$R = \frac{1}{hA_{fin}\eta_{fin}}$$

\chapter{Transient Temperature Changes}
\section{Lump Analysis}
Treating the body like it just has one internal temperature that is the same throughout. Used when $Bi = \frac{hL_c}{k} = \leq 0.1$, $L_c = V/A_s$. 
$$hA_s(T_\infty - T) dt = mc_p dT$$
$$\frac{T(t) - T_\infty}{T_i - T_\infty} = \exp(-bt)$$ 
Where $b = \frac{hA_s}{\rho V c_p}$

$$Q_{net} = mc_p[T(t) - T_i]$$

\section{Transient Heat Conduction}
We study 1-D transient heat conduction for walls, spheres, and cylinders.
\begin{itemize}
\item $\theta = \frac{T - T_\infty}{T_i-T_\infty}$ = dimensionless temperature
\item $X = x/L$ = dimensionless time
\item $Bi = hL/k$ = dimensionless heat transfer coefficient
\end{itemize}

The fourier series for each of these is on the formula sheet. When $\tau > 0.2$ we can use the single-term fourier approximation. 
$\tau = \frac{\alpha t}{L^2}$.
\subsection{Illustrative Example: }
\paragraph{How long until the centre of an egg is 70 Celsius? }
\begin{enumerate}
\item $Bi = \frac{hr_0}{k}$. $Bi > 0.1 \to $ Transient 1-D analysis. Less means lump analysis. 
\item Find $\lambda_1$ and $A_1$ from the table.
\item Find $\tau$, the fourier number. $$\frac{T_0-T_\infty}{T_i - T_\infty} = A_1 e^{-\lambda_1^2 \tau}$$ to determine the fourier number $\tau$. 
$T_0$ is the final temperature. 
\item $t = \frac{\tau r_0^2}{\alpha}$ = time until middle reaches final temperature $T_0$. 
\end{enumerate}


\chapter{External Forced Convection}
\section{Physical Mechanisms of Convection}
\paragraph{No Slip Condition: } Fluid velocity is zero and `sticks' to the surface of objects it flows by. Therefore, we have pure conduction 
at the surface of objects in fluids. 

IMPORTANT NUMBERS: 
\begin{enumerate}
\item \textbf{Nusslet} $Nu = \frac{h L_c}{k} = \frac{\dot{q}_{conv}}{\dot{q}_{cond}}$
\item 
\item 
\end{enumerate}

\section{Classifying Fluid Flows}
\begin{itemize}
\item Viscous: internal stickiness is significant. Inviscous: internal stickiness is not relevant. 
\item \textbf{External} means flow is unbounded. \textbf{Internal} means flow is bounded on all sides (e.g. pipe).
\item \textbf{Compressible} corresponds to $Ma > 0.3$ or $v > 100m/s$. 
\item \textbf{Natural} means that it's only stuff like buoyancy and convection.
\item \textbf{Steady} means stable through time. \textbf{Uniform} means stable with location. 
\item \textbf{Transient} means developing steady flows. \textbf{Periodic} means oscillatory.  
\end{itemize}

\section{Velocity Boundary Layer}
Consider fluid flowing over a surface. Due to the non-slip condition, the fluid at the surface doesn't move. 
$\delta$ is the height above the surface at which the x-velocity reaches $0.99v$ where $v$ is the upstream velocity. It's the
\textbf{boundary layer thickness}. It divides the `boundary layer' from the `irrotational flow' area. 

$F/A = \tau$ which is shear stress. $$\tau_s = \mu \frac{du}{dy}|_{y=0}$$

$\nu = \mu/\rho$ is kinematic viscosity. Measured in stokes (cm2/s)

More practically: $$\tau_s = C_f\frac{\rho V^2}{2}$$


\chapter{In-Class Things and Guides}

\section{Thermal Boundary Layer}
This is the same as the velocity boundary layer but for temperature. It's where $T_s - T = 0.99(T_\infty - T_s)$. Its size 
increases in the direction of flow. $\dot{Q}_x$ depends strongly on temperature gradient at the surface, so it's important!

\textbf{Prantl Number} is the relative thickness of the velocity : thermal layer. $$Pr = \frac{\nu}{\alpha} = \frac{\mu c_p}{k}$$

\section{Laminar and Turbulent Flows}

Reynold numbers dictate whether you are in laminar, turbulent, or transitional flow. If $Re > Re_{cr}$, we have turbulent flow.

\section{Drag and Heat Transfer in External Flow}

Lift is just the sum of forces that are perpendicular to flow. 

$$C_D = \frac{F_D}{\frac{1}{2}\rho V^2 A}$$

There is a pressure component and a frictional component to drag force. Pressure dominates with large Reynold's numbers while friction 
dominates with small Reynold's numbers. 

The area in that formula for drag coefficient is the frontal projection of the object for most objects, but is the actual area of a 
flat plate that is in line with the flow. 

\subsection{Heat Transfer}
Heat transfer relies on the same variables as drag. $$Nu = C_{onst} Re_i^m Pr^n$$ where $m, n$ are constant integer exponents and $C$ is 
a constant that results from the geometry. 

Since $C_D, h$ vary along the surface, we want the AVERAGE values of them. 

$x_cr$ is the x coordinate where flow becomes turbulent over a plate. $Re_{cr} = 5 \times 10^5$ is the generally 
accepted way to get the critical reynold's number. 

\paragraph{Friction Coefficients at $x$:}
\begin{itemize}
\item Laminar: $$\delta_{v, x} = \frac{4.91 x}{Re_x^0.5}$$
$$C_{fx} = \frac{0.664}{Re_{x}^0.5}$$
$$Nu_x = \frac{h_xx}{k} = 0.332 Re_x^{0.5}Pr^{1/3}$$
\item Turbulent: $$\delta_{v, x} = \frac{0.38 x}{Re_x^{1/5}}$$
$$C_{fx} = \frac{0.059}{Re_x^{1/5}}$$
$$Nu_x = \frac{h_xx}{k} = 0.0296Re_x^{0.8}Pr^{1/3}$$
\end{itemize}

\paragraph{Average Friction Coefficients: }
\begin{itemize}
\item Laminar: $$C_f = \frac{1.33}{Re_L^{0.5}}$$
\item Turbulent: $$C_f = \frac{0.074}{Re_L^{1/5}}$$
\item Laminar AND Turbulent: $$C_f = \frac{0.74}{Re_L^{1/5}}$$
\end{itemize}

\textbf{Rough, Turbulent Surfaces: } $$C_f = (1.89 - 1.62 \log \frac{\varepsilon}{L})^{-2.5}$$

\paragraph{Average Heat Transfer Coefficients}
\begin{itemize}
\item Laminar: $Nu_{avg} = \frac{hL}{k} = 0.664 Re_L^0.5 Pr^{1/3}$
\item Turbulent: $Nu_{avg} = \frac{hL}{k} = 0.037 Re_L^{0.8}Pr^{1/3}$
\item Entire Plate: $Nu_{avg} = \frac{hL}{k} = (0.037 Re_l^{0.8} - 871)Pr^{1/3}$
\end{itemize}
There's some substantial deviation from these if the fluid has very niche characteristics (small Pr for liquid metals, etc.). Here is a 
formula that applies to all fluids: $$Nu_x = \frac{h_xx}{k} = \frac{0.3387 Pr^{1/3} Re_x^{0.5}}{[1+(0.0468/Pr)^{2/3}]^{1/4}}$$

\subsection{Unheated Starting Length}
It's common for the first bit of the plate that the fluid goes over to not be heated. Let $\Xi$ be the distance from the physical and heated starting points.
\begin{itemize}
\item Laminar: $$Nu_x = \frac{Nu_{x, (\Xi = 0)}}{[1-(\Xi/x)^{3/4}]^{1/3}}$$
$$h_{avg} = \frac{2[1-(\Xi/x)^{3/4}]}{1-\Xi/L} h_{x=L}$$

\item Turbulent: $$Nu_x = \frac{Nu_{x, (\Xi = 0)}}{[1-(\Xi/x)^{9/10}]^{1/9}}$$
$$h_{avg} = \frac{5[1-(\Xi/x)^{9/10}]}{4(1-\Xi/L)}h_{x=L}$$
\end{itemize}

\subsection{Uniform Heat Flux}
When the plate has uniform heat flux instead of uniform heat. Relations for $\Xi \neq 0$ apply here too. 
\begin{itemize}
\item Laminar: $$Nu_x = 0.453 Re_x^{0.5}Pr^{1/3}$$
\item Turbulent: $$Nu_x = 0.0308 Re_x^{0.8}Pr^{1/3}$$
\end{itemize}

$$\dot{Q} = \dot{q}_s A$$ $$\dot{q_s} = h_x[T_s(x) T_\infty] \to T_s(x) = T_\infty + \frac{\dot{q}_s}{h_x}$$


\end{document}