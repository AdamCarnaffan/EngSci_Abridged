\documentclass[a4paper,12pt]{report}
\usepackage{color}
\usepackage{graphicx}
\usepackage{subfig}
\usepackage{listings}
\usepackage{media9}
\usepackage{hyperref}
\usepackage{amssymb}
\usepackage{mathtools} 

% \def\reals{{\rm I\!R}}
\def\reals{\mathbb{R}}
\def\integers{\mathbb{Z}}


\newtheorem{theorem}{Theorem}

\definecolor{dkgreen}{rgb}{0,0.6,0}
\definecolor{gray}{rgb}{0.5,0.5,0.5}
\definecolor{mauve}{rgb}{0.58,0,0.82}

\begin{document}

\title{ECE355: Signal Analysis and Communications}
\author{Aman Bhargava}
\date{September-December 2020}
\maketitle

\tableofcontents

\section{Introduction and Course Information}

This document offers an overview of the ECE355 course. They comprise my condensed course notes for the course. No promises are made relating to the correctness or completeness of the course notes. These notes are meant to highlight difficult concepts and explain them simply, not to comprehensively review the entire course.

Primary course topics include:

\begin{enumerate}
\item Signals and Systems (Chapter 2).
\item Frequency Domain Analysis (Chapters 3-5).
\item Sampling (Chapter 9).
\item Introduction to Communication Systems (Chapter 8).
\end{enumerate}

\paragraph{Course Information}
\begin{itemize}
\item Professor: Ben Liang
\item Course: Engineering Science, Machine Intelligence Option
\item Term: 2020 Fall
\end{itemize}




\chapter{Signal Basics}

\section{Definitions}

\paragraph{Two types of signals: } Continuous ($f(x)$ defined $\forall x \in \reals$) and Discrete $f(n)$ defined $\forall n \in \integers$.

\paragraph{Power and Energy of a signal: }
\begin{itemize}
	\item Power of $x(t)$ is $|x(t)|^2$.
	\item Energy of $x(t)$ is defined on interval $[t_1, t_2]$ as 
	$$E_{[t_1, t_2]} = \int_{t_1}^{t_2} |x(t)|^2 dt$$
	$$E_{n_1 \leq n \leq n_2} = \sum_{n=n_1}^{n_2} |x[n]|^2$$

	\item Average power of in $[t_1, t_2]$:
	$$P_{[t_1, t_2]} = \frac{E_{[t_1, t_2]}}{t_2 - t_1}$$
	$$P_{[n_1, n_2]} = \frac{E_{n_1 \leq n \leq n_2}}{n_2 - n_1 + 1}$$
	\item Total Energy: 
	$$E_{\infty} = \lim_{T\to\infty} \int_{-T}^T |x(t)|^2 dt$$
	$$E_{\infty} = \lim_{N\to\infty} \sum_{n = -N}^{N} |x[n]|^2$$
\end{itemize}


\section{Signal Transformations}

\paragraph{Time Shifting: } \textit{Shifts $t_0$ units RIGHT}
$$y(t) = x(t-t_0)$$
$$y[n] = x[n-n_0]$$

\paragraph{Time Scaling: } \textit{Speeds original signal up by factor $a$)} (or slowed down by factor $\frac{1}{a}$). Time reversal occurs when $a<0$.
$$y(t) = x(at)$$

\paragraph{Continuous Scaling AND Shifting: } It is important to remember the following steps for $y(t) = x(at+b)$

\begin{enumerate}
\item \textbf{SHIFT}: $v(t) = x(t+b)$.
\item \textbf{SCALE}: $y(t) = v(at)$.
\end{enumerate}

\paragraph{Discrete Time Scaling AND Shifting: } Remember to IGNORE fractional indexes. Interpolation for `slowing down' a signal is a poorly defined process that will be covered later.

\section{Periodic Signals}

\paragraph{Definition: } A signal is periodic iff $\exists T > 0$ s.t. $x(t+T) = x(t)$ $\forall t\in\reals$. 

\begin{itemize}
\item $T$ is the period of the signal.
\item \textbf{Fundamental} period is the smallest possible $T$.
\item If $x(t)$ is constant, then the fundamental period is undefined.
\end{itemize}

\section{Even and Odd Signals}

\paragraph{Even: } $x(t) = x(-t)$
\paragraph{Odd: } $x(-t) = -x(t)$

\textbf{ANY SIGNAL} can be decomposed into an even and odd component.

$$x_{even}(t) = \frac{1}{2} (x(t) + x(-t))$$
$$x_{odd}(t) = \frac{1}{2} (x(t) - x(-t))$$

$$x(t) = x_{even}(t) + x_{odd}(t)$$


\section{Complex Exponential}
\paragraph{Function Family: } $x(t) = c e^{at}$, $c, a\in \mathbb{C}$

\subsection{Complex Number Review}
\begin{itemize}
\item $z = a+jb$, $z\in\mathbb{C}$, $a, b \in \mathbb{R}$, $j = \sqrt{-1}$.
\item Magnitude = $r=|z| = \sqrt{a^2 + b^2}$.
\item Angle (phase) = $\theta = \arctan(\frac{b}{a})$.
\item $z = re^{j\theta} = r(\cos\theta + j\sin\theta)$.
\end{itemize}

\subsection{Useful Sinusoid Shortcuts}
$$\cos\theta = \frac{e^{j\theta} + e^{-j\theta}}{2}$$
$$\sin\theta = \frac{e^{j\theta} - e^{-j\theta}}{j2}$$

\subsection{Periodic Case}
Letting $c = 1$:
$$x(t) = e^{j\omega_0 t} = cos(\omega_0 t) + j\sin(\omega_0 t)$$
\begin{itemize}
\item Combination of two \textbf{real} signals.
\item $|x(t)| = 1$ $\forall t\in\reals$.
\item \textbf{PERIOD: } $T=\frac{2\pi}{|\omega_0|}$.
\end{itemize}


For more general $c\in\mathbb{C}$: \textit{We let $c = |c|e^{j\phi}$} where $\phi$ is the phase. Then $x(t) = ce^{j\omega_0 t}$, then: 

$$x(t) = |c|e^{j(\omega_0 t + \phi)}$$

For fully general $c, a \in \mathbb{C}$: $x(t) = ce^{at} = ce^{(r+j\omega_0)t}$ where $a = (r+j\omega_0)$

$$x(t) = |c|e^{rt} e^{j(\omega_0 t + \phi)}$$
$$Re\{x(t)\} = |c|e^{rt} \cos(\omega_0 t + \phi)$$

Which leads to two cases (`forced harmonic' when $r>0$, `damped harmonic' when $r < 0$).

\subsection{Discrete Time Complex Exponential} 

$$x[n] = e^{j\omega_0 n} = \cos(\omega_0 n) + j\sin(\omega_0 n)$$

\begin{itemize}
\item Signal `hops' around the unit circle in \textbf{increments of} $\omega_0$.
\item \textbf{NOT ALWAYS PERIODIC!} $\omega_0 \in a2\pi, a\in \mathbb{Q}$ for periodicity to hold.
\item $c\in\mathbb{C}$ just changes magnitude and phase.
\end{itemize}

\section{Unit Step and Impulse}

\subsection{Discrete Time}

\begin{equation}
	u[n] = 
	\begin{cases}
		0 & n < 0 \\
		1 & n \geq 0 \\
	\end{cases}
\end{equation}

\begin{equation}
	\delta[n] = 
	\begin{cases}
		0 & n \neq 0 \\
		1 & n = 0 \\
	\end{cases}
\end{equation}


\paragraph{Important Properties: } 
\begin{itemize}
\item $\delta[n] = u[n] - u[n-1]$
\item $u[n] = \sum_{k=0}^\infty \delta[n-k]$
\item $u[n] = \sum_{m=n}^{-\infty} \delta[m]$, if we let $m = n-k$
\item $u[n] = \sum_{m=-\infty}^n \delta[m]$, if we let $m = n-k$
\end{itemize}

\textbf{Sampling property: } $x[n] \delta[n-n_0] = x[n_0] \delta[n-n_0]$

\subsection{Continuous Time Case}

\begin{equation}
u(t) = 
\begin{cases}
	0 & t < 0 \\
	1 & t > 0 \\
\end{cases}
\end{equation}

\begin{equation}
\delta(t) = \frac{d}{dt} u(t)
\end{equation}

There are some more formal definitions, but this will do for now. Consider it a finite amount of energy in an infinitely small period. 

\paragraph{Important Properties: } 

\begin{itemize}
\item $\int_{\infty} \delta(t) dt = 1 = u(\infty) - u(-\infty)$
\item $u(t) = \int_{\infty}^{t} \delta(\tau) d\tau$
\item $u(t) = \int_{0}^{\infty} \delta(\sigma - t) d\sigma$
\item Sampling still holds: $x(t_0) = \int_{\infty} x(t) \sigma(t-t_0) dt$
\end{itemize}

\section{Basic System Properties}

\begin{enumerate}
\item \textbf{Memoryless} if $y(t_0)$ depends ONLY on $x(t_0)$ $\forall t_0$.
\item \textbf{Invertable} when you can recover input SIGNAL given output SIGNAL (not value).
\item \textbf{Causal}:
\begin{itemize}
\item \textit{Discrete Time: } If $y[n_0]$ does not depend on \textit{future information}.
\item \textit{Continuous Time: } If $y(t_0)$ does not depend on $x(t)$ for $t \geq t_0$.
\item \textbf{Equivalently: } If two inputs are identical for $t < t_0$, the outputs are identical for $t < t_0$.
\end{itemize}
\item \textbf{Stability: } Small input does not lead to infinite output. \textit{Definition: } System is \textbf{bounded-input-bounded-output} (BIBO) if bounded input leads to bounded output.
\item \textbf{Time Invariance: } Shifted input $\to$ shifted output with same time shift. If $y(t) = sin(x(t))$, then for input $x(t-t_0)$, the output is: 
\begin{equation}
\begin{split}
y(t) &= sin(x(t-t_0)) \\
&= y(t-t_0)
\end{split}
\end{equation}
\item \textbf{Linearity: } Must satisfy \textit{additivity} and \textit{homogineity} (in other words: \textbf{superposition}).
$$ax_1(t) + bx_2(t) \to_S ay_1(t) + by_t(t)$$
Where $x_1 \to y_1$, $x_2 \to y_2$.
\begin{itemize}
\item Linear systems commute with scaling and addition. 
\item Scaling then pushing through system is the same as pushing through system and scaling. 
\item Adding signals together then pushing through the system is the same as pushing each individual signal through the system and then adding the outputs together.
\end{itemize}
\end{enumerate}


\paragraph{Initial Rest Condition: } if $x(t) = 0$ $\forall t < t_0$, then the corresponding output $y(t) = 0$ $\forall t< t_0$.


\chapter{Linear Time-Invariant Systems}

\section{Discrete Time LTI Properties}

The system's response to an impulse function $\delta[n]$ is called $h[n]$. \textbf{IF WE KNOW $h[n]$}, we can map any input to its respective output due to \textit{time invariance and linearity}!

\begin{equation}
y[n] = \sum_{k=-\infty}^{\infty} x[k] h[n-k]
\end{equation}

\subsection{Convolution in Discrete Time}

\paragraph{Convolution of $x[n], h[n] \to y[n]$ } is defined as: 
\begin{equation}
y[n] = \sum_{k=-\infty}^{\infty} x[k] h[n-k] \equiv x[n] * h[n]
\end{equation}


Interpretations:
\begin{enumerate}
\item Weighted superposition of time shifted impulse responses (pretty clear).
\item Sum over dimension $k$ of function $x[k]h[n-k]$.
\begin{itemize}
\item $h[n-k] = h[-k+n]$ is \textbf{flipped} and shifted \textbf{right} by $n$.
\item Multiply $h[n-k]$ by $x[k]$ and \textbf{sum} the result to get $y[n]$.
\end{itemize}
\end{enumerate}


\section{Continuous Time LTI Systems}

\paragraph{Unit Impulse Response: } $h(t)$ results from input $\delta(t)$. 

\begin{theorem}
\begin{equation}
\begin{split}
y(t) = \int_{-\infty}^{\infty} x(\tau)h(t-\tau) d\tau \\
y(t) \equiv x(t) * h(t)
\end{split}
\end{equation}
\end{theorem}



\section{Properties of LTI Systems}

\paragraph{Properties of Convolution: } 
\begin{enumerate}
\item \textbf{Convolution is Commutative: } $x(t)*h(t) = h(t)*x(t)$
\item \textbf{Convolution is Associative: } $x(t)*[h_1(t)*h_2(t)] = [x(t)*h_1(t)]*h_2(t)$
\item \textbf{Convolution is Distributive: } $x(t)*[h_1(t) + h_2(t)] = x(t)*h_1(t) + x(t)*h_2(t)$
\end{enumerate}

\paragraph{Identity System: } If the unit impulse response is $\delta(t)$, then the system is the \textbf{identity system} -- it will produce the same output as the input.

\paragraph{Time shift note: } Shifting the input AND the step response yields double that shift. One must only shift one to shift the output correspondingly (time invariance property).

\paragraph{Properties of LTI: } 
\begin{enumerate}
\item \textbf{Memory: } LTI is \textit{memoryless} iff $h(t) = K\delta(t)$ ($K\in\reals$), leading to $y(t) = Kx(t)$ being the \textbf{only memoryless LTI} family.
\item \textbf{Invertibility: } If an LTI is invertible, \textit{its inverse is also an LTI}. 
\item \textbf{Causality: } LTI is causal iff $h(t) = 0\,\,\,\forall t < 0$. \textit{Note that `causality' is interchangable for `initial rest condition'}.
\item \textbf{Stability: } Conditions for LTI stability are as follows (tl;dr: bounded unit impulse response is the necessary and sufficient condition). 
\begin{itemize}
\item \textit{Absolutely Integrable: } $\int_{-\infty}^\infty |h(\tau)|d\tau < \infty$
\item \textit{Absolutely Summable: } $\sum_{n=-\infty}^\infty |h[n]| < \infty$
\end{itemize}
\end{enumerate}


\section{Linear Constant-Coefficient Differential (Difference) Equations: LCCDE's}

\paragraph{General Form for Continuous Time: } 

\begin{equation}
\sum_{k=0}^N a_k \frac{d^k}{dt^k} y(t) = \sum_{k=0}^{M} b_k \frac{d^k}{dt^k} x(t) 
\end{equation}

\begin{itemize}
\item Always assume \textit{initial rest} condition.
\item LCCDE's are a subset of \textit{causal LTI systems}.
\item LCCDE's provide a \textit{close approximation} of most LTI systems. \textit{This is because the ``transfer function'' is a rational function that can easily approximate most functions}.
\end{itemize}

\paragraph{Standard Solution: } We would normally use \textbf{method of undetermined coefficients} -- this is a relatively unsophisticated method, but it's important to keep in mind. 

\begin{enumerate}
\item We are given some $x(t)$ and are asked to solve for $y(t)$ given an LCCDE relationship (e.g. $A\frac{d}{dt}y(t) + B y(t) = x(t)$).
\item Assume a solution $y(t) = y_h(t) + y_p(t)$ where 
\begin{itemize} 
\item $y_h$ is the solution for the case of $x(t) = 0\,\,\forall t$. This is the \textit{natural response} or the \textit{unforced response}.
\item $y_p$ is the solution for the given $x(t)$.
\end{itemize}

\item We guess $y_h(t)$ is of the form $Ae^{st}$. We can substitute into the homogenous equation to solve a relationship between $A, s$ and initial conditions.
\item For $y_p(t)$ we guess again (usually the same form as $x(t)$). We substitute in and solve for coefficients. 
\item Finally, we solve for remaining unknown coefficients given some initial conditions.
\end{enumerate}


\paragraph{Better tools for solving LTI/LCCDES's: } Solve them in the frequency domain using the Fourier transform and Laplace transform! 

\paragraph{How to solve LTI via Transfer Function Laplace Transform} 
\textbf{Example: } $\frac{d}{dt} y(t) + 4y(t) = x(t)$. 

\begin{enumerate}
\item Assume there exists some $H(s)$ (eigenvalue of the eigenfunction family $e^{st}$). 
\item Therefore, $H(s) [\frac{d}{dt} e^{st} = 4e^{st}] = e^{st}$
\item We can now solve for $H(s) = \frac{1}{s+4}$
\item If we can write the input $x(t)$ as the sum of complex exponentials, we can solve for $y(t)$!
\item For the case $x(t) = \cos(\pi t) = 0.5 e^{j\pi t} + 0.5 e^{-j\pi t}$, the corresponding output would be: 
\begin{equation}
y(t) = \frac{1}{2} \frac{1}{(j\pi)+4} e^{(j\pi)t} + \frac{1}{2} \frac{1}{-j\pi+4} e^{-j\pi t}
\end{equation}
\end{enumerate}




\section{Discrete Case for LCCDE's}

\textit{Stands for Linear Constant Coefficient Difference Equations}: 

\paragraph{General form for discrete LCCDE: } 

\begin{equation}
\hat{a}_0 y[n] + \sum_{k=1}^N \hat{a}_n (y[n] - y[n-k]) = \hat{b}_0 x[n] + \sum_{k=1}^M \hat{b}_n(x[n] - x[n-k])
\sum_{k=0}^N a_k y[n-k] = \sum_{k=0}^M b_k x[n-k]
\end{equation}

\paragraph{Solution Options: } 
\begin{enumerate}
\item We could solve in a manner similar to the CT case: 
\begin{equation}
y[n] = y_h[n] + y_p[n]
\end{equation}

\item That said, there is a \textbf{more efficient} method that takes advantage of the discrete nature of this problem. By rearranging, we get: 
\begin{equation}
y[n] = \frac{1}{a_0} [\sum_{k=0}^M b_k x[n-k] - \sum_{k=1}^N a_k y[n-k]]
\end{equation}
\end{enumerate}



\chapter{Fourier Representations of Periodic Signals}

\section{LTI Response to Complex Exponentials}

It turns out that the guess of $y_p(t) = Ae^{st}$ is \textbf{always} a good guess for LTI systems. In fact, it is \textbf{SCALED} every time! Given a system with an impulse reponse $h(t)$ that is fed an input of $e^{st}$: 

\begin{equation}
\begin{split}
y(t) &= \int_{-\infty}^{\infty} h(\tau) e^{s(t-\tau)} d\tau \\
&= e^{st} \int_{-\infty}^\infty h(\tau) e^{-s\tau} d\tau \\
&= e^{st} H(s)
\end{split}
\end{equation}

Where $H(s) \equiv \int_{-\infty}^\infty h(\tau) e^{-s\tau} d\tau$ is the \textbf{Laplace Transform} of $h$ (a.k.a. the \textbf{transfer function}).

\begin{theorem}
If $x(t) = e^{st}$ and $H(s)$ exists: 
\begin{equation}
y(t) = H(s)e^{st}
\end{equation}
In essence, the response of the LTI is a scaled version of the same complex exponential by factor $H(s)$ defined above.
\end{theorem}

\subsection{Discrete Time Case}

Let $e^{sn} \equiv z^n$. $z^n \to h[n] \to y[n]$. We define

\begin{equation}
H(z) = \sum_{k=-\infty}^{\infty} h[k] z^{-k}
\end{equation}
As the \textbf{$z$-transform}. 

\begin{theorem}
$y[n] = H[z]z^n$. 
\begin{itemize}
\item $z^n$ is therefore an \textbf{eigenfunction} of any LTI.
\item $H[z]$ is the corresponding \textbf{eigenvalue} of that eigenfunction.
\end{itemize}
\end{theorem}


\paragraph{Cautionary Notes: } 
\begin{itemize}
\item $(e^x)^z$ does not hold in general for $x, y, \in \mathbb{C}$.
\item $(e^x)^n$ DOES hold for $n\in \mathbb{Z}$.
\end{itemize}

\section{Continuous Time Fourier Series}

\paragraph{Guiding Fact: } Almost all periodic signals are approximated by a sum of weighted \textbf{harmonically related} complex exponentials.

\begin{equation}
x(t) = \sum_{k=-\infty}^{\infty} a_k e^{jk\omega_0 t}
\end{equation}

\paragraph{Note on utilizing frequency response with Fourier series signal representation: } If $x(t) = \sum_{k}^{} a_k e^{jk\omega_0 t}$: 

\begin{equation}
y(t) = \sum_{k=-\infty}^{\infty} a_k H(jk\omega_0) e^{jk\omega_0 t}
\end{equation}

\paragraph{Finding Laplace Transform for LTI: } If given the system in implicit form: \textbf{input the eigenfunction} $x(t) = e^{st}$

\paragraph{Result of Passing Signal through LTI: } 
\begin{enumerate}
\item Frequency-dependent \textbf{amplification} 
\item Frequency-dependent \textbf{phase shift} 
\end{enumerate}

\subsection{Calculating CT Fourier Series}

\begin{equation}
a_k = \frac{1}{T} \int_{T} x(t) e^{-jk\omega_0 t} dt
\end{equation}

Where $\int_{T}$ is integration over any period of length $T$.



\subsection{Convergence of Fourier Series}

\paragraph{Which periodic signals have Fourier series representations? } 

\begin{theorem}
We define the finite fourier series 
\begin{equation}
x_n(t) = \sum_{k=-N}^{N} a_k e^{j\omega_0 kt}
\end{equation}
where the error is $e_N(t) = x(t) - x_N(t)$, $E_N \int_{T}^{} |e_n(t)|^2 dt$. 

\textbf{If $x(t)$} has finite energy in one period, then 
\begin{equation}
\lim_{N\to \infty} E_N = 0
\end{equation}

\end{theorem}

If $x(t)$ satisfies \textbf{Dirchlet conditions} (nearly all signals), then $x(t) = FS(x(t))$ except at \textbf{isolated points}.

\paragraph{Gibbs Phenomena: } Small oscillations about discontinuities in a signal (e.g. approximations of a square wave).

\section{Properties of Continuous Time Fourier Series}

\paragraph{Setup: } Let $x(t)$ be periodic with fundamental period $T\to \omega_0 = \frac{2\pi}{T}$ that has Fourier series coefficients $a_k$.

\paragraph{Properties: } 

\begin{enumerate}
\item \textbf{Linearity: } 
\begin{equation}
Ax(t) + By(t) \underrightarrow{\,\,\mathcal{F}\,\,} Aa_k + Bb_k
\end{equation}

\item \textbf{Time Shift: } 
\begin{equation}
x(t-t_0) \underrightarrow{\,\,\mathcal{F}\,\,} e^{-jk\omega_0 t_0} a_k
\end{equation}


\item \textbf{Time Scaling: } For $\alpha > 0$: 
\begin{equation}
x(\alpha t) \underrightarrow{\,\,\mathcal{F}\,\,} a_k
\end{equation}

Where $a_k$ now has fundamental period $\alpha \omega_0$

\item \textbf{Time Reversal: } 
\begin{equation}
x(t) \underrightarrow{\,\,\mathcal{F}\,\,} a_{-k}
\end{equation}
Therefore even functions have $a_k = a_{-k}$ and odd functions have $-a_k = a_{-k}$.

\item \textbf{Conjugation: } 
\begin{equation}
x^* (t) \underrightarrow{\,\,\mathcal{F}\,\,} a_{-k}^*
\end{equation}
\begin{itemize}
\item Special case: $x(t)$ is a real signal so $x^*(t) = x(t)$. The we have $a_k = a_{-k}^*$
\item Known as ``conjugate symmetric'' or ``Hermitian''.
\item If you know $x(t)$ is real, then: $$x(t) = a_0 \sum_{k=1}^{\infty} 2 A_k \cos (k\omega_0 t + \theta)$$
\textit{Where} $a_k = A_k e^{j\theta_k}$.
\end{itemize}

\item \textbf{multiplication: } Multiplication in one domain corresponds to convolution in the other!
\begin{equation}
x(t) y(t) \underrightarrow{\,\,\mathcal{F}\,\,} c_k = \sum_{l = -\infty}^{\infty} a_l b_{k-l}
\end{equation}

\item \textbf{Parsevel's Transform: } The average power of $x(t)$ is given by: 

\begin{equation}
\frac{1}{T} \int_{T}^{} |x(t)|^2 dt = \sum_{k=-\infty}^{\infty} |a_k|^2
\end{equation}

\end{enumerate}
\textit{Other properties can be found in table 3.1}

\section{Continuous Time Fourier Transform}

\textit{Recap: } Fourier series can approximate nearly all periodic signals. we now introduce the Fourier Transform, a system to approximate aperiodic signals!

\begin{theorem}

We define the Fourier Transform as follows: 

\begin{equation}
X(j\omega) = \int_{-\infty}^{\infty} x(t) e^{-j\omega t} dt
\end{equation}

And its inverse operation as: 

\begin{equation}
x(t) = \frac{1}{2\pi} \int_{-\infty}^{\infty} X(j\omega) e^{j\omega t} d\omega
\end{equation}
\end{theorem}

\paragraph{Important Notes on the Fourier Transform Properties: } 

\begin{itemize}
\item $\text{sinc}(x) \equiv \frac{\sin(\pi x)}{\pi x}$
\item Wider signal in time domain leads to narrower signal in frequency domain.
\item $x(t) = \delta(t) \to X(j\omega) = 1$
\item $X(j\omega) = 2\pi \delta(\omega) \to x(t) = 1$
\item $x(t) = u(t) \to X(j\omega) = \delta(\omega) \pi + \frac{1}{j\omega}$
\end{itemize}

\subsection{Periodic Signal Fourier Transform}

\paragraph{Key struggle: } Periodic signals have infinite energy and therefore do not converge in the Fourier transform integral. 

\begin{theorem}
For an arbitrary periodic $x(t) = \sum_{k = -\infty}^{\infty} a_k e^{jk\omega_0 t}$:
\begin{equation}
X(j\omega) = \sum_{k = -\infty}^{\infty} 2\pi a_k \delta(\omega - k\omega_0)
\end{equation}

In other words, a periodic signal is simply a collection of delta functions in the frequency domain. 
\end{theorem}

Steps to find the Fourier Transform of periodic signal: 
\begin{enumerate}
\item Find Fourier Series version of $x(t)$.
\item Use $a_k$ in the above formula $X(j\omega) = \sum_{k=-\infty}^{\infty} 2\pi a_k \delta(\omega - k\omega_0)$.
\end{enumerate}



\subsection{Properties of Continuous Time Fourier Transform}

\begin{enumerate}
\item \textbf{Linearity: } \begin{equation}
ax(t) + by(t) \to aX(j\omega) + bY(j\omega)
\end{equation}

\item \textbf{Time Shift: } \begin{equation}
x(t-t_0) \to e^{-j\omega t_0} X(j\omega)
\end{equation}

\item \textbf{Time + Frequency Scaling: } \begin{equation}
x(at) \to \frac{1}{|a|} X(\frac{j\omega}{a})
\end{equation}

\item \textbf{Conjugation: } \begin{equation}
x^*(t) \to X^*(-j\omega)
\end{equation}

\item \textbf{Differentiation and Integration: } \begin{equation}
x'(t) \to j\omega X(j\omega) \\
\int_{-\infty}^{t} x(\tau) d\tau \to \frac{1}{j\omega} X(j\omega)
\end{equation}

\item \textbf{Duality: } If $x(t) \to X(j\omega) = g(\omega)$ then $g(t) \to 2\pi x(-\omega)$.

\item \textbf{Frequency Shifting: } \begin{equation}
e^{j\omega_0 t} x(t) \to X(j(\omega-\omega_0))
\end{equation}


\item \textbf{Differentiation in Frequency Domain: } \begin{equation}
-jtx(t) \to \frac{d}{d\omega} X(j\omega)
\end{equation}



\item \textbf{Integration in Frequency Domain: } \begin{equation}
\frac{-1}{j\omega} x(t) + \pi x(0) \delta(t) \to \int_{-\infty}^{\omega} X(j\eta) d\eta
\end{equation}

\item \textbf{Parseval's Relation: } \begin{equation}
\int_{-\infty}^{\infty} |x(t)|^2 dt = \frac{1}{2} \int_{-\infty}^{\infty} |X(j\omega)|^2 dw
\end{equation}

\end{enumerate}



































\end{document}

