\documentclass[a4paper,12pt]{report}
\usepackage{color}
\usepackage{graphicx}
\usepackage{subfig}
\usepackage{listings}
\usepackage{media9}
\usepackage{hyperref}
\usepackage{amssymb}
\usepackage{mathtools} 
\usepackage{amsmath}
\usepackage{extarrows} 

% \def\reals{{\rm I\!R}}
\def\reals{\mathbb{R}}
\def\integers{\mathbb{Z}}
\def\fft{\xlongleftrightarrow{\mathcal{F}}}
\def\fs{\xlongleftrightarrow{\mathcal{FS}}}


\newtheorem{theorem}{Theorem}

\definecolor{dkgreen}{rgb}{0,0.6,0}
\definecolor{gray}{rgb}{0.5,0.5,0.5}
\definecolor{mauve}{rgb}{0.58,0,0.82}

\begin{document}

\title{ECE368: Probabilistic Reasoning}
\author{Aman Bhargava}
\date{January-April 2021}
\maketitle

\tableofcontents

\section{Introduction and Course Information}

\paragraph{Course Information}
\begin{itemize}
\item Professors: Prof. Saeideh Parsaei Fard and Prof. Foad Sohrabi
\item Course: Engineering Science, Machine Intelligence Option
\item Term: 2021 Winter
\end{itemize}

\paragraph{Main Course Topics} 
\begin{itemize}
\item Vector, temporal, and spatial models.
\item Classification and regression model training.
\item Bayesian statistics, frequentist statistics.
\end{itemize}




\chapter{Review Topics}

\textit{See ECE286 notes for further reference} 

\section{Review of Probability Functions}

\paragraph{Probability Mass Function: } For \textit{discrete random variables}, $P_X(x)$denotes the probability that random variable $X$ takes on value $x$.

\paragraph{Probability Density Function: } For \textit{continuous random variables}, the probability $\Pr\{X\in [x_1, x_2]\}$ is given by $\int_{x_1}^{x_2} f_X(x) dx$.

Joint PMF's and PDF's are similarly defined. 

\paragraph{Marginal Probability Distributions: } Given joing PMF $P_{X, Y}(x, y)$ or PDF $f_{X,Y}(x, y)$, we can \textbf{marginalize} them as follows:
\begin{equation}
P_X(x) = \sum_{y\in Y}^{} P_{X,Y}(x, y)
\end{equation}
\begin{equation}
f_X(x) = \int_{-\infty}^{\infty} f_{X,Y}(x,y)dy
\end{equation}

\paragraph{Conditional Probability Functions: } 
\begin{equation}
P_{Y|X}(y, x) = \frac{P_{X,Y}(x,y)}{P_X(x)} 
\end{equation}


\paragraph{Prior Probability: } Probability \textbf{before} an additional observation is made (hence \textit{prior}). Example: $P_X(x)$.

\paragraph{Posterior Probability: } Probability \textbf{after} an observation is made (hence \textit{post}erior). Example: $P_{X|Y}(x, y)$.

\paragraph{Bayes Rule: }
\begin{equation}
P(B|A) = P(A|B)\frac{P(B)}{P(A)}
\end{equation}

\section{Expectation, Correlation, and Independence} 

\paragraph{Expectation Value: } $\mathbb E[x] = \sum_{x\in X}^{} P_X(x) = \int_{-\infty}^{\infty} xf_X(x) dx$

\paragraph{Law or Large Numbers: } $\lim_{N\to\infty} \frac{1}{N} \sum_{i=1}^{N} x_i = \mathbb E[X]$

\paragraph{Variance: } 

\begin{equation}
\begin{split}
\text{Var} (X) &= \mathbb E[(X-\mathbb E[x])^2] \\
&= \mathbb E[X^2] - (\mathbb E[X])^2
\end{split}
\end{equation}

\paragraph{Covariance: } 
\begin{equation}
\begin{split}
\text{Cov}(X,Y) &= \mathbb E[(X-\mathbb E[X]) (Y-\mathbb E[Y])] \\
&= \mathbb E_{XY}[XY] - \mathbb E[X]\mathbb E[Y]
\end{split}
\end{equation}

\paragraph{Correlation Coefficient: } 
\begin{equation}
\rho_{XY} = \frac{\text{Cov}(X,Y)}{\sqrt{\text{Var}(X)}\sqrt{\text{Var}(Y)}} 
\end{equation}
\begin{itemize}
\item $\rho_{XY}\in [-1, 1]$
\item $\rho > 0$ indicates positive correlation (line of best fit has positive slope).
\item $\rho < 0$ indicates negative correlation.
\item \textbf{$\mathbb E[XY] = \mathbb E[X] \mathbb E[Y]$ iff $X,Y$ are uncorrelated}.
\end{itemize}

\paragraph{Independence} 
\begin{theorem}{Independence}
Random variables $X,Y$ are independent \textbf{iff} 
\begin{equation}
P_{XY}(x,y) = P_X(x) \cdot P_Y(y)
\end{equation}
This also means that $\rho_{XY} = 0$, $P(X|Y) = P(X)$, etc.
\end{theorem}



\section{Laws of Large Numbers}

\paragraph{Weak Law: } Sample mean converges to the mean.

\paragraph{Strong Law: } If $\{x_i\}$ are \textbf{independent, identically distributed} (i.i.d.) random variables with mean $\mu$, then the \textbf{probability of} the sample mean = $\mu$ is 1 as $n\to \infty$.

\chapter{Parameter Estimation}

\section{Estimation Terminology}

























\end{document}
